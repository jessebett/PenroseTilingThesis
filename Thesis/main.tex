% book example for classicthesis.sty
\documentclass[
  % Replace twoside with oneside if you are printing your thesis on a single side
  % of the paper, or for viewing on screen.
  %oneside,
  oneside,
  11pt, a4paper,
  footinclude=true,
  headinclude=true,
  cleardoublepage=empty
]{scrbook}

\tolerance=100
\usepackage{lipsum}
\usepackage[hidelinks]{hyperref}
\usepackage{graphicx}
\usepackage[linedheaders,parts,pdfspacing]{classicthesis}
\usepackage{acronym}
\usepackage{microtype}
\usepackage[british]{babel}

\usepackage{float}
\usepackage{caption}
\usepackage{subcaption}
\usepackage{gensymb}

\usepackage{amsmath,amsfonts,amsthm,amssymb}
\usepackage[cal=cm]{mathalfa}

\newtheorem{mydef}{Definition}
\newtheorem{myprop}{Property}
\newtheorem{mythm}{Theorem}
\newtheorem{mylem}{Lemma}

\graphicspath{{./Figures/}}


\title{Penrose Aperiodic Tiling of the Plane and Graphical Geodesics}
\author{Jesse Bettencourt\\1144386\\[80pt]  \textit{Supervisor: Dr. Miroslav Lovric}\\[30pt]ISCI 4A12}

\begin{document}

\maketitle

%%*******************************************************
% Abstract
%*******************************************************
\pdfbookmark[1]{Abstract}{Abstract}
\chapter*{Abstract}

In this thesis we will first provide historical background and introduce foundational concepts to the study of aperiodic plane tilings, namely, Penrose tilings. We will explore three methods of Penrose tiling construction: inflation and deflation, Up-Down generation, and Pentagrid duals. The features of the Penrose tiling are surprising, and challenge previous notions in tiling theory. We will explore these features, proving the uncountability of Penrose tilings, the admittance of five-fold symmetries, local isomorphism between all Penrose tilings, and the surprising ratio of prototiles. Finally, we will introduce and describe graphical geodesics on the Penrose tiling, and characterize them according to their global meandering.
%%%*******************************************************
% Dedication
%*******************************************************
\thispagestyle{empty}
\pdfbookmark[1]{Dedication}{Dedication}

\vspace*{3cm}

\begin{center}
To Alex, Mack, Maia, and Rachel.\\
Who are unlikely to read this,\\
but by living with me know its contents.
\end{center}

\vfill

\begin{figure}
\centering
\includegraphics[width=\textwidth]{FrontStar}
\end{figure}


%%*******************************************************
% Acknowledgments
%*******************************************************
\pdfbookmark[1]{Acknowledgements}{acknowledgements}
\chapter*{Acknowledgements}
First and foremost I'd like to thank my supervisor Dr. Miroslav Lovric for his guidance on this project. I am extremely grateful for this opportunity to self-direct my research questions and pursue a project through curiosity, intrigue, and a bit of aestheticism. Miroslav enabled all of this through his supervision. But I'd also like to thank him for his influence far before the supervision of this project when, as the instructor for first year math in Integrated Science 1A24, he introduced me to Mathematics and irrevocably altered the direction of my subsequent studies. Thank you for introducing me to Mathematics as it is, an art.\\
I would also like to acknowledge my peers and faculty in the Integrated Science program, who have taught me how to ask deep questions about the universe. And, to my peers and faculty in the Mathematics Department, who have taught me how to begin answering them.
%%*******************************************************
% Declaration
%*******************************************************
\pdfbookmark[0]{Technical Note}{Technical Note}
\chapter*{Technical Note}
\thispagestyle{empty}

A significant effort in this project was the creation of graphics illustrating the various features of Penrose tilings. In doing so, a Mathematica library for the creation and visualization of Penrose tilings was written. The majority of the figures found in this thesis were produced with this library. Figures not produced for this thesis will be otherwise noted. The Mathematica library, including all code to generate this document, the figures within, and an accompanying Beamer presentation, is available on a GitHub repository. Click the logo to visit the repository.

\begin{center}
\href{https://github.com/jessebett/PenroseTilingThesis}{\includegraphics[width=0.2\textwidth]{githublogo}}
\end{center}

%%*******************************************************
% Table of Contents
%*******************************************************
\pdfbookmark[1]{\contentsname}{tableofcontents}

\setcounter{tocdepth}{2} % <-- 2 includes up to subsections in the ToC
\setcounter{secnumdepth}{3} % <-- 3 numbers up to subsubsections

\tableofcontents 

%*******************************************************
% List of Figures and of the Tables
%*******************************************************

%*******************************************************
% List of Figures
%*******************************************************    
\pdfbookmark[1]{\listfigurename}{lof}
\listoffigures





\chapter{Introduction to Tessellations}
\section{Definitions} %DONE 
We will begin our discussion of Penrose tilings with some definitions given by Senechal \cite{Senechal1996}. First, a formal definition of a tiling of Euclidean n-space, $\mathbb{E}^n$:



\begin{mydef}
A \textbf{tiling} $\mathcal{T}$ of the space $\mathbb{E}^n$ is a countable family of closed sets, $T$, called tiles:
\begin{equation*}
\mathcal{T}=\{T_1,T_2,\mathellipsis\}
\end{equation*}
such that
\begin{enumerate}
\item $\mathcal{T}$ has no overlaps: $\mathring{T_i} \cap \mathring{T_j}=\emptyset$ if $i\neq j$
\item $\mathcal{T}$ has no gaps: $\bigcup_{i=1}^\infty T_i = \mathbb{E}^n$
\end{enumerate}
\end{mydef}


Here $\mathring{T}$ denotes the interior of tile $T$. Further, we assume that a tile is the closure of its interior, and that tiles have positive volume. These assumptions allow, for example, a line segment to be a tile in $\mathbb{E}^1$ but not in $\mathbb{E}^2$. Notice that this definition of tiling neither restricts the shape of the individual tiles nor the number of unique tiling shapes.  

However, we will impose some restrictions on the tiles here. Since we are considering tilings of the plane, our tiling must satisfy the above definition for $n=2$, that is, for the Euclidean plane $\mathbb{E}^2$. Further, while the general definition does not impose any criteria on tile shape, for our purposes we will only consider n-dimensional polytopes. In the case of the plane, we consider only 2-polytopes, or polygons. To denote the facets of our tiles we will use \textbf{edges} to denote the 1-dimensional faces, and \textbf{vertices} to denote the 0-dimensional faces. 

\begin{mydef}
Let $\{T_1,T_2,\mathellipsis\}$ be the set of tiles of tiling $\mathcal{T}$, partitioned into a set of equivalence classes by criterion $\mathcal{M}$. The set, $\mathcal{P}$, of representatives of these equivalence classes is called the \textbf{protoset} for $\mathcal{T}$ with respect to $\mathcal{M}$.
\end{mydef}

For example, consider an infinite black-and-white checkerboard (Fig.\ref{fig:checkerboard}). Each tile in the checkerboard is either a black or white square, and the tiling is given by the matching rule that black squares may only share edges with white squares, and vice-versa. In this example, the equivalence class criterion, $\mathcal{M}$, is the colour of the square together with this matching rule. The protoset, $\mathcal{P}$, for the checkerboard is the set containing two elements: a black square and a white square.

\begin{figure}[H]

  \hspace*{\fill}
\begin{subfigure}[t]{0.4\textwidth}
	\includegraphics[width=\textwidth]{Checkerboard}
    \caption{Checkerboard}
    \label{fig:checkerboard}
\end{subfigure}
  \hspace*{\fill}
\begin{subfigure}[t]{0.4\textwidth}
	\includegraphics[width=\textwidth]{CheckerboardOther}
    \caption{Different Criterion.}
    \label{fig:critchecker}
\end{subfigure}
  \hspace*{\fill}\\
  \hspace*{\fill}
  \begin{subfigure}[t]{0.4\textwidth}
  	\includegraphics[width=\textwidth]{CheckerboardCircles}
      \caption{Deformed Checkerboard}
      \label{fig:deformedchecker}
  \end{subfigure}
      \hspace*{\fill} 
\begin{subfigure}[t]{0.4\textwidth}
	\includegraphics[width=\textwidth]{CheckerboardAperiodic}
    \caption{Sub-Periodic Checkerboard}
    \label{fig:aperiodicchecker}
\end{subfigure} 
\hspace*{\fill}
\label{fig:checker}
\caption{A protoset of black and white squares can admit multiple tilings. Some of them may be periodic, but this is not necessary.}
\end{figure}

\begin{mydef}
If $\mathcal{T}$ is a tiling with protoset $\mathcal{P}$, then we say that $\mathcal{P}$ admits $\mathcal{T}$.
\end{mydef}

In the checkerboard example we say that the protoset containing black and white squares, together with the matching criterion, admits a checkerboard tiling (Fig.\ref{fig:checkerboard}). It is important to note that a protoset can admit multiple tilings given different criterion. For example, the checkerboard protoset also admits the tiling in Fig.\ref{fig:critchecker} under a different matching rule. 

Further, it is also worth noting that abstract criterion, such as colour-matching or directed edges, can be accomplished instead through deformation of the edges of protoset tiles such that the criterion is forced. Consider again the criterion for producing the checkerboard tiling (Fig.\ref{fig:checkerboard}), that black squares may only share edges with white squares, and vice-versa. One way to realize this criterion is to deform all black squares into circles, and to deform white squares into asteroids. As seen in Fig.\ref{fig:deformedchecker}, these deformations force a tiling such that all white tiles occupy the interstices of the black circles. In other words, the protoset of deformed tiles, the black circle and white asteroid, can only admit one unique tiling. We are able to force a checkerboard pattern without any abstract criterion such as colour-matching adjacent tiles. While it is important to understand that abstract matching criterion can be realized by edge deformations, considering simpler protosets with more complicated criterion will usually allow greater insight into the structure and patterns within a tiling than would considering a simple criterion with complicated prototiles. 

\begin{mydef}
A tiling of  $\mathbb{E}^n$ is said to be \textbf{periodic} if it admits translational symmetry in $n$ linearly independent directions.
\end{mydef}

For example, the checkerboard tiling in Fig.\ref{fig:checkerboard} is periodic, as it has translational symmetry in two directions: vertical translation by two units, and horizontal translation by two units. Likewise, we can see that the criterion generating Fig.\ref{fig:critchecker} produces a tiling which is also periodic, with vertical and horizontal translational symmetry given by translation by three units. 

A corollary of a tiling being periodic is that it can be generated by translating a finite subregion by linear combinations of the independent directions. For example, the checkerboard can be generated by infinitely repeating a $2\times2$ subregion translated vertically and horizontally. 

\begin{mydef}
A tiling of  $\mathbb{E}^n$ is said to be \textbf{sub-periodic} if it admits translational symmetry in $k$ linearly independent directions, where $1\leq k\leq n$. A tiling is said to be \textbf{non-periodic} if it admits no translational symmetry. 
\end{mydef}

For example, consider the checkerboard pattern with a single `column' translated vertically by a half-unit distance (Fig.\ref{fig:aperiodicchecker}). This tiling admits translational symmetry in the vertical direction with translation by two units. However, we can see that there is no horizontal symmetry because horizontal translation will not allow the shifted column to overlap anywhere. For this reason, there are also no diagonal translational symmetries admitted. This column shifted checkerboard is sub-periodic. 

Unlike periodic tilings, neither sub-periodic nor non-periodic tilings can be generated by translations of a finite subregion. 

We've seen that the protoset of black and white squares admits multiple types of tiling, both periodic and sub-periodic. Some protosets, such as the circle and asteroid (Fig.\ref{fig:deformedchecker}), only admit tilings which are periodic. A question which motivated the discovery of the Penrose tiles was whether we can find protosets that admit \textit{only} non-periodic tilings.

\begin{mydef}
A protoset $\mathcal{P}$ is said to be \textbf{aperiodic} if it admits only non-periodic tilings. A tiling $\mathcal{T}$ from an aperiodic protoset is called an \textbf{aperiodic tiling}. 
\end{mydef}

Here, `only' means that no subset of the protoset, nor the entire protoset, can tile periodically. 


\section{Kepler's Monsters} %START
\section{Discovering Aperiodic Tessellations} %DONE
Studied in the mid-1900's was a class of tilings produced by spiraling deformed isosceles triangles about a central point. A method discovered by Micheal Goldberg in 1955 generates these spiral tilings readily. Further, due to their spiraling arrangement, these tilings admit no translational symmetries, so non-periodic. However, these spiral tilings are admitted by protosets which also admits periodic tilings. In fact, all other non-periodic tilings known at the time, also notably Golomb's `reptiles', were formed from protosets which also admit periodic tilings.

\begin{figure}
\begin{subfigure}{0.4\textwidth}
\includegraphics[width=\textwidth]{monohedral}
\end{subfigure}\hfill
\begin{subfigure}{0.4\textwidth}
\includegraphics[width=\textwidth]{monohedral-trans}
\end{subfigure}
\caption{Spiral tilings are non-periodic, but their protosets also admit periodic tilings \cite{Austin2007}.}
\end{figure}

It was conjectured that no protosets existed which allow only non-periodic tiling. That is, that no aperiodic protosets were possible. This was proved untrue when, in 1961, Hao Wang began writing on protosets composed of unit squares with coloured edges, called Wang dominoes. The matching rules for these dominoes require adjacent dominoes to share the same colour on their abutting edge, with rotation and reflection of the prototiles not allowed. Wang was concerned with questions in logic, specifically investigating algorithmic procedures for deciding whether any arbitrary set of dominoes would admit a plane tiling, called decision procedures. Wang posed a conjecture that any protoset which can admit a plane tiling will also admit a periodic tiling, and Wang showed that if this is true then there must also be an algorithmic procedure for deciding whether such a set will tile. 

In 1964, Robert Berger showed in his doctoral thesis from Harvard University that Wang's conjecture is false, that there is no algorithmic decision procedure for the tiling ability of any arbitrary protoset. The implication of Berger's work is that there must be an aperiodic protoset of Wang dominoes. Further, Berger constructed such a protoset, using 20,426 dominoes. In his later work, Berger produced a smaller protoset, requiring only 104 dominoes. Donald Knuth further reduced this number by finding an aperiodic Wang protoset of only 92 tiles.

As discussed previously, tiling matching rules can be realized by deformations on the edges of the prototiles. For instance, instead of the coloured edges of the Wang dominoes, consider deforming the edges of the squares into protrusions and cavities, producing prototiles like pieces of a jigsaw puzzle. These deformations can be chosen in such a way that they permit only the same arrangements as did the coloured edges of the dominoes, however, this representation also allows for rotation and reflection of the prototiles, previously not considered by Wang. 

By representing the protoset as described above, in 1971 Raphael Robinson constructed an aperiodic protoset requiring only six prototiles. The prototiles used by Robinson are considered to still be in the category of `square' tiles, and whether size of an aperiodic protoset of square tiles can be reduced to fewer than six tiles is still unknown. However, it is conjectured that six is the minimum number of square tiles to produce an aperiodic protoset. This was supported in 1977 when Robert Ammann found a different aperiodic protoset of six square prototiles.

In 1973, Roger Penrose found an aperiodic protoset of six, non-square tiles.  Penrose continued this process of reduction and, in 1971, reduced the size of his aperiodic protoset to four tiles, and shortly after reduced them further to two tiles. There are arbitrarily many representations of these two prototiles, but two representations are considered simplest and most recognizable. 

The first of these representations, the Penrose Rhombs, is a set of two rhombi with equal lengths, but different internal angles (See Fig.\ref{fig:Rhombs}). The thin rhombus, \textbf{t}, has corner angles of $\frac{\pi}{5}$ and $\frac{4 \pi}{5}$ (or $36 \degree$ and $144 \degree$). The thick rhombus, \textbf{T}, has corner angles  $\frac{2 \pi}{5}$ and $\frac{3 \pi}{5}$ (or $72 \degree$ and $108 \degree$). Both thin and thick rhombi can be produced by the composition of Robinson's Golden triangles, named for Robinson's 1975 notes on their construction. The rhombs can be produced by reflecting the isosceles Robinson triangles across their bases. The ratio of the leg and base lengths in the Robinson triangles is the Golden Ratio, \textbf{ $\phi= \frac{1+\sqrt{5}}{2}\approx 1.6180$}. This ratio will appear repeatedly throughout this discussion of Penrose tilings. 

The other common representation of Penrose's tiles is the Kite and Dart protoset, named by John Conway. These prototiles are also produced by the composition of Robinson triangles, rather by reflection across their legs. 

Neither the Rhombs nor the Kites and Darts tilings were publicized until after Penrose filed for patents on the shapes, due to their commercial potential as board game puzzles. It was Martin Gardner who finally published Penrose and Conway's work on describing these tilings, as well as their method of constructing the tiling by substitution, in his 1977 ``Mathematical Games'' column in \textit{Scientific American}. 

In 1981, Dutch mathematician Nicolaas de Bruijn provided other methods of constructing the Penrose tiling, by projecting of five-dimensional cubic structure and also by infinite paths on directed graphs, which will discussed later. De Bruijn also showed the relationship between the Penrose tiling and five families of parallel lines, called the Pentagrid. De Bruijn's work represents the characterization of Penrose tilings as algebraic objects, and offers deep insight into their nature. 

The practical importance of these theories on aperiodic tiling was realized in the 1980's with the discovery by materials scientist Dan Shechtman of certain aluminum-manganese alloys which produce crystallographic diffraction patterns with no translational symmetries. Shechtman was awarded the 2011 Nobel Prize in Chemistry for his discovery of these structures, known now as quasicrystals. In 2009, the first natural quasicrystal was discovered, the mineral icosahedrite, whose diffraction pattern displays fivefold symmetry previously thought impossible in crystalline structures. This past month, in March 2015, the second natural quasicrystal was discovered in a meteorite, with a diffraction pattern displaying decagonal symmetry \cite{Bindi2015}. Marjorie Senechal formalizes the relationship between aperiodic tiling and the geometry of quasicrystals in her 1995 monograph \textit{Quasicrystals and Geometry}. 


\chapter{Constructing the Penrose Tiling}

\section{Rhombus Representation of the Penrose Tiling}
As introduced, there are multiple protoset representations that admit the Penrose tiling. However, the construction methods outlined in this section were described initially by de Bruijn, who favoured the Rhombs representation.  For this reason, we will consider only construction of the Penrose tiling by the Rhombs protoset (see Fig.\ref{fig:Rhombs}). 

Since this protoset is composed of two rhombi, it might be suspected that the prototiles can be arranged to produce a periodic tiling, since rhombi can generally produce periodic tilings. Consider, for example, the tiling produced by adjacent rows of thick and thin rhombi (see Fig.\ref{fig:RhombsPeriodic}). This tiling underscores the necessity of edge-matching rules for the aperiodic protoset. 

The edge-matching rules of the Penrose rhombs have been illustrated with blue and orange curves in Fig.\ref{fig:Rhombs}. If two tiles are adjacent in a tiling, then curves of the same colour must meet along the abutting edge. For example, the previously discussed periodic rhombus tiling is not permitted given this edge-matching criteria, because these curves do not meet along adjacent tiles (see Fig.\ref{fig:RhombsPeriodicRules}). Further, recall from previous description that matching criteria can be realized through prototile edge deformations (see Fig.\ref{fig:deformedchecker}). This allows us to consider the Rhombs protoset with matching rules as distinct from the Rhombs without matching rules, permitting us to describe the former as aperiodic. 


\begin{figure}[H]
\begin{subfigure}[t]{0.4\textwidth}
\centering
\includegraphics[width=\textwidth]{RhombPeriodic}
\caption{Periodic tiling of thin and thick rhombi, admitted in the absence of edge-matching rules.}
\label{fig:RhombsPeriodic}
\end{subfigure}\hfill
\begin{subfigure}[t]{0.4\textwidth}
\centering
\includegraphics[width=\textwidth]{RhombPeriodicRules}
\caption{Matching rules require blue and orange curves to meet along adjacent edges. Clearly, this tiling does not satisfy these matching rules.}
\label{fig:RhombsPeriodicRules}
\end{subfigure}
\caption{Periodic tiling using Penrose rhombs is only possible in the absence of edge-matching rules.}

\end{figure}

\begin{figure}
        \centering
        \begin{subfigure}[b]{\textwidth}
        \begin{subfigure}[b]{0.5\textwidth}
                \centering
                \includegraphics[scale=0.5]{RobinsonFat}
                \subcaption{Thick Robinson Triangle}
                \label{fig:RobThick}
        \end{subfigure}\hfill%
        ~ %add desired spacing between images, e. g. ~, \quad, \qquad, \hfill etc.
          %(or a blank line to force the subfigure onto a new line)
        \begin{subfigure}[b]{0.5\textwidth}
                \centering
                \includegraphics[scale=0.3]{RobinsonSkinny}
                \subcaption{Thin Robinson Triangle}
                \label{fig:RobThin}
        \end{subfigure}
        \caption*{Robinson Triangles with Golden leg-to-base ratio, $\phi$.}
        \label{fig:RobTris}
        \end{subfigure}
        ~ %add desired spacing between images, e. g. ~, \quad, \qquad, \hfill etc.
          %(or a blank line to force the subfigure onto a new line)
          
        \begin{subfigure}[b]{\textwidth}
        \begin{subfigure}[b]{0.5\textwidth}
        \centering
                \raisebox{40px}{\includegraphics[scale=0.5]{RhombFat}}
                \caption{Thick Penrose Rhomb}
                \label{fig:ThickRhomb}
        \end{subfigure}\hfill
        \begin{subfigure}[b]{0.5\textwidth}
        \centering
                \includegraphics[scale=0.6]{RhombSkinny}
                \caption{Thin Penrose Rhomb}
                \label{fig:ThinRhomb}
        \end{subfigure}   
                \caption*{ \centering Penrose Rhombs from Robinson triangles. Blue and Orange curves illustrate matching rules.}
                \label{fig:JustRhombs}     
        \end{subfigure}
        \caption{Penrose Rhombs are generated by reflecting the Robinson triangles across their bases. The Rhombs with an edge-matching criterion, illustrated by the blue and orange curves, form an aperiodic protoset.}
        \label{fig:Rhombs}
\end{figure}

Now that we have seen the Penrose rhombs protoset, and the criteria which allow two tiles to meet, we can investigate how to construct valid tilings. 


\section{Non-Local Growth}
Penrose tilings cannot be procedurally constructed one-tile-at-a-time, as one would tile a floor. In attempting to construct a tiling this way, there is no guarantee that any finite, valid arrangement of tiles can be continued infinitely. 


 that the fact that simply because a protoset admits a tiling does not guarantee that any valid arrangement of those prototiles can be continued to an infinite tiling. 
The Penrose tiling features a notable difficulty in procedurally constructing tilings one tile at a time due to the non-locality of tiling growth, this will be discussed shortly. Though, to motivate the consideration of Penrose tiling despite the frustration of non-locality, I will provide a theorem now which will be proven later.

\begin{mythm}
The Penrose Rhombs protoset, with matching rules, admits a tiling of the plane.
\end{mythm}

\section{Substitution Construction}
Despite non-locality, Penrose and Conway showed that we can construct arbitrarily large tilings by repeated application of substitution rules.

To clearly understand the substitution rules on the rhombs, we will first demonstrate the substitution of the constituent Robinson triangles (Fig.\ref{fig:RobSubs})

\begin{figure}[H]
        \centering
        \begin{subfigure}[t]{\textwidth}
        \begin{subfigure}[t]{0.4\textwidth}
                \centering
               \raisebox{9px}{\includegraphics[scale=0.4]{RobinsonFat}}
        \end{subfigure}\hfill \raisebox{30px}{\huge$\rightarrow$} \hfill%
        ~ %add desired spacing between images, e. g. ~, \quad, \qquad, \hfill etc.
          %(or a blank line to force the subfigure onto a new line)
        \begin{subfigure}[t]{0.4\textwidth}
                \centering
                \includegraphics[scale=0.4]{RobFatSub}
        \end{subfigure}
        \caption{Substitution rule for Thick Robinson triangle}
        \label{fig:RobSubThick}
        \end{subfigure}
        ~ %add desired spacing between images, e. g. ~, \quad, \qquad, \hfill etc.
          %(or a blank line to force the subfigure onto a new line)
          
        \begin{subfigure}[b]{\textwidth}
        \begin{subfigure}[b]{0.4\textwidth}
        \centering
		\includegraphics[scale=0.3]{RobinsonSkinny}
        \end{subfigure}\hfill \raisebox{30px}{\huge$\rightarrow$} \hfill
        \begin{subfigure}[b]{0.4\textwidth}
        \centering
                \includegraphics[scale=0.3]{RobSkinnySub}
        \end{subfigure}   
        \caption{Substitution rule for Thick Robinson triangle}
        \label{fig:RobSubThin}
        \end{subfigure}
        \caption{Substitution rules demonstrated on Robinson triangles. The substitution fragments each Robinson triangle into Robinson sub-triangles. The vertices of these sub-triangles are given such that the edges of the substituted triangle are divided according to the Golden ratio. The ratios of these divided edges are labeled accordingly. Note that these substitutions produce valid arrangements as per the edge-matching rules, illustrated by the orange and blue curves.}
        \label{fig:RobSubs}
\end{figure}

Recall that these Robinson triangles generate Penrose Rhombs by reflection across their bases (Fig.\ref{fig:Rhombs}). With that in mind, consider the above substitution applied to the Rhombs, such that each constituent Robinson triangle composing the Rhomb undergoes substitution as above. These substitution rules, applied to the Penrose Rhombs, is known as the process of \textbf{deflation}.

\begin{figure}[H]
        \centering
        \begin{subfigure}[t]{\textwidth}
        \begin{subfigure}[t]{0.4\textwidth}
                \centering
               \raisebox{30px}{\includegraphics[scale=0.4]{RhombFat}}
        \end{subfigure}\hfill \raisebox{75px}{\huge$\rightarrow$} \hfill%
        ~ %add desired spacing between images, e. g. ~, \quad, \qquad, \hfill etc.
          %(or a blank line to force the subfigure onto a new line)
        \begin{subfigure}[t]{0.4\textwidth}
                \centering
                \includegraphics[scale=0.4]{RhombFatSub}
        \end{subfigure}
        \caption{Deflation rule for Thick Penrose Rhomb}
        \label{fig:RhombSubThick}
        \end{subfigure}
        ~ %add desired spacing between images, e. g. ~, \quad, \qquad, \hfill etc.
          %(or a blank line to force the subfigure onto a new line)
          
        \begin{subfigure}[b]{\textwidth}
        \begin{subfigure}[b]{0.4\textwidth}
        \centering
		\includegraphics[scale=0.4]{RhombSkinny}
        \end{subfigure}\hfill \raisebox{75px}{\huge$\rightarrow$} \hfill
        \begin{subfigure}[b]{0.4\textwidth}
        \centering
                \includegraphics[scale=0.4]{RhombSkinnySub}
        \end{subfigure}   
        \caption{Deflation rule for Thin Penrose Rhomb}
        \label{fig:RobSubThin}
        \end{subfigure}
        \caption{Deflation rules of the Penrose Rhombs. Pre-substituted rhomb is illustrated with dashed black lines. Note that these substitutions produce valid arrangements as per the edge-matching rules, illustrated by the orange and blue curves.}
        \label{fig:RobSubs}
\end{figure}

This substitution process can be used to generate arbitrarily large Penrose tilings through a complementary process, called \textbf{inflation}. Following a series of deflation substitutions, the generated tiling is then `inflated' such that the deflated rhombuses are scaled to the same size as the original, pre-deflated rhombuses. Notice that a single application of the deflation rule scales the composite rhombi by a factor of $\frac{1}{\phi}$. This means that, after $n$ iterations of the deflation process, the resulting rhombs can be scaled, or inflated, by a factor of $\phi^n$ to return them to the original rhombus size. We can see in Figures \ref{fig:ThickDeflate} and \ref{fig:ThinDeflate} that iterations of the deflation process produce increasingly complex patterns. 

\begin{figure}[H]
\centering
\begin{tabular}{cc}
	        \begin{subfigure}[b]{0.4\textwidth}
	        \centering
	        \raisebox{30px}{\includegraphics[scale=0.4]{FatInflation0}}
	        \end{subfigure}   &
            \begin{subfigure}[b]{0.4\textwidth}
             \centering
             \includegraphics[scale=0.4]{FatInflation1}
             \end{subfigure}   \\
	       	 \begin{subfigure}[b]{0.4\textwidth}
             \centering
             \includegraphics[scale=0.4]{FatInflation2}
             \end{subfigure}   &
             \begin{subfigure}[b]{0.4\textwidth}
             \centering
             \includegraphics[scale=0.4]{FatInflation3}
             \end{subfigure}   \\
             \begin{subfigure}[b]{0.4\textwidth}
             \centering
             \includegraphics[scale=0.4]{FatInflation4}
             \end{subfigure}   &
             \begin{subfigure}[b]{0.4\textwidth}
             \centering
             \includegraphics[scale=0.4]{FatInflation5}
             \end{subfigure}   \\
\end{tabular}
\caption{Five successive iterations of the deflation process applied to the Thick Rhombus.}
\label{fig:ThickDeflate}
	\end{figure}

\begin{figure}[H]
\centering
\begin{tabular}{cc}
	        \begin{subfigure}[b]{0.4\textwidth}
	        \centering
	        \raisebox{0px}{\includegraphics[scale=0.4]{SkinnyInflation0}}
	        \end{subfigure}   &
            \begin{subfigure}[b]{0.4\textwidth}
             \centering
             \includegraphics[scale=0.4]{SkinnyInflation1}
             \end{subfigure}   \\
	       	 \begin{subfigure}[b]{0.4\textwidth}
             \centering
             \includegraphics[scale=0.4]{SkinnyInflation2}
             \end{subfigure}   &
             \begin{subfigure}[b]{0.4\textwidth}
             \centering
             \includegraphics[scale=0.4]{SkinnyInflation3}
             \end{subfigure}   \\
             \begin{subfigure}[b]{0.4\textwidth}
             \centering
             \includegraphics[scale=0.4]{SkinnyInflation4}
             \end{subfigure}   &
             \begin{subfigure}[b]{0.4\textwidth}
             \centering
             \includegraphics[scale=0.4]{SkinnyInflation5}
             \end{subfigure}   \\
\end{tabular}
\caption{Five successive iterations of the deflation process applied to the Thin Rhombus.}
\label{fig:ThinDeflate}
	\end{figure}
	
Consider again that under inflation each of the thick and thin rhombi in the deflated patterns will be scaled back to the original rhombi size. Further, consider that this substitution process can be iterated arbitrarily many times, generating a tiling that is arbitrarily large. 

Also, notice that even after successive applications of the deflation substitution, the resulting tiling is still valid under the initial matching rules. Again, in Figures \ref{fig:ThickDeflate} and \ref{fig:ThinDeflate} this is illustrated by the continuity of the orange and blue edge-matching curves.

\subsection{Extension Theorem}

\section{Up-Down Construction}

In his 1990 paper \textit{Updown generation of Penrose patterns}, de Bruijn describes a construction method attributed to Conway by which tilings are generated by infinite sequences of steps along a directed graph. The Up-Down generation process borrows some concepts from the substitution method, namely that we build tilings by substituting tiles with constituent decomposition tiles, as per substitution rules. The key difference, however, is that we build a tiling instead by considering the relationship between a constituent tile and its inflation. As the name suggests, Up-Down generation constructs tilings through two processes: Up and Down.

In introducing this construction method, we must consider again the Robinson triangles. As with the substitution method, decomposition of these triangles is dependent on their orientation. Unlike substitution, the influence of triangle orientation under the Up-Down method is subtle and not automatic. Moving forward, we must be explicit with the triangle orientation, see Fig.\ref{fig:OrientedRob}. Further, recall from the substitution method that oriented Robinson triangles, arranged according to those matching rules, will determine a Penrose tiling of the plane by Penrose Rhombs. Robinson triangles can be transformed into Penrose Rhombs by reflection about their base, see Fig.\ref{fig:Rhombs}. As such, while the Up-Down method describes the process to construct a tiling of the plane by the oriented Robinson triangles of Fig.\ref{fig:OrientedRob}, we are also indirectly constructing a Penrose tiling by Rhombs. The relationship between these representations will be discussed in greater detail later, with the discussion of mutual local derivability. 

 

\begin{figure}[h]
        \begin{subfigure}[t]{0.4\textwidth}
                \includegraphics[width=\textwidth]{TFRob}
                \caption*{\Large T}
        \end{subfigure}\hfill
        \begin{subfigure}[t]{0.3\textwidth}
                \includegraphics[width=\textwidth]{TSRob}
                \caption*{\Large t}
        \end{subfigure}\\
        
        \begin{subfigure}[t]{0.4\textwidth}
                \raisebox{0px}{\includegraphics[width=\textwidth]{T'FRob}}
                \caption*{\Large T'}
        \end{subfigure}\hfill
        \begin{subfigure}[t]{0.3\textwidth}
                \includegraphics[width=\textwidth]{T'SRob}
                \caption*{\Large t'}
        \end{subfigure}  
        \caption{Oriented Robinson triangles. Thick or Obtuse Robinson Triangle is denoted with T. Thin or acute Robinson triangle is denoted with t. Reversed orientation is denoted with '. Arrowed edges illustrate Up-Down matching rules.}  
        \label{fig:OrientedRob}                    
\end{figure}

The matching rules under the substitution method behave slightly differently than the matching rules in the Up-Down method, illustrated with arrowed edges in Fig.\ref{fig:OrientedRob}. Here, the rules will indicate type and direction of edges, which must agree along abutting tiles. Additionally, the edge type and direction will satisfy decomposition rules according to the relationship between a triangle and it's constituent triangles, see Fig.\ref{fig:EdgeRules}. 

\begin{figure}[H]
        \begin{subfigure}[t]{0.4\textwidth}
                \includegraphics[width=\textwidth]{a1}
        \end{subfigure}\hfill
        \begin{subfigure}[t]{0.4\textwidth}
                \includegraphics[width=\textwidth]{a1inflate}
        \end{subfigure}\\
        
        \begin{subfigure}[t]{0.4\textwidth}
                \includegraphics[width=\textwidth]{a2}
        \end{subfigure}\hfill
        \begin{subfigure}[t]{0.4\textwidth}
                \includegraphics[width=\textwidth]{a2inflate}
        \end{subfigure}\\    
        
        \begin{subfigure}[t]{0.4\textwidth}
                \includegraphics[width=\textwidth]{a3}
        \end{subfigure}\hfill
        \begin{subfigure}[t]{0.4\textwidth}
                \includegraphics[width=\textwidth]{a3inflate}
        \end{subfigure}\\                           
        
        \begin{subfigure}[t]{0.4\textwidth}
                \includegraphics[width=\textwidth]{a4}
        \end{subfigure}\hfill
        \begin{subfigure}[t]{0.4\textwidth}
                \includegraphics[width=\textwidth]{a4inflate}
        \end{subfigure}\\   
        \caption{Parent edges (Left) and their constituent edges (Right)}     
        \label{fig:EdgeRules}
\end{figure} 


\pagebreak
Consider again the substitution rules on the Robinson triangles described in Fig.\ref{fig:RobSubs}. This substitution is also employed in the Up-Down method. However, in this case we will label the constituent triangles of the substitution. We do this so we can explicitly describe the relationship between a constituent triangle and its parent triangle. We label this relationship using mapping functions, given by lowercase Greek letters: $\alpha, \beta, \delta, \gamma$. We see the oriented Robinson triangles, substituted with relationship labels, in Fig.\ref{fig:OrientedSub}.

These relationship labels can be considered as mappings from a constituent triangle to its parent triangle. These mappings are as follows:
\begin{align}
\epsilon &: T \rightarrow T  &\epsilon' &: T' \rightarrow T' \nonumber\\  
\alpha &: T \rightarrow t  &\alpha' &: T' \rightarrow t' \nonumber\\  
\beta &: t \rightarrow t  &\beta' &: t' \rightarrow t' \label{eq:maps}\\  
\delta &: t' \rightarrow T  &\delta' &: t \rightarrow T' \nonumber\\ 
\gamma &: T' \rightarrow T  &\gamma' &: T \rightarrow T' \nonumber
\end{align}
Again, these mappings are also labeled in Fig.\ref{fig:OrientedSub}. For an example of how to interpret this mapping, consider the mapping $\delta$ which maps constituent \textbf{t'} to parent \textbf{T}. This tells us that following relationship $\delta$, the yellow \textbf{t'} triangle will be located inside the pink \textbf{T} parent.



\begin{figure}[H]
        \begin{subfigure}[t]{0.55\textwidth}
                \includegraphics[width=\textwidth]{TF}
        \end{subfigure}\hfill
        \begin{subfigure}[t]{0.35\textwidth}
                \includegraphics[width=\textwidth]{TS}
        \end{subfigure}\\
        
        \begin{subfigure}[t]{0.55\textwidth}
                \raisebox{55px}{\includegraphics[width=\textwidth]{TF'}}
        \end{subfigure}\hfill
        \begin{subfigure}[t]{0.35\textwidth}
                \includegraphics[width=\textwidth]{TS'}
        \end{subfigure}   
        \caption{Oriented Robinson triangles with their constituent triangles. The relationship mappings between a constituent and parent triangles are given by lowercase Greek letters. For example, constituent triangle \textbf{T'} is related to parent triangle \textbf{T} under the mapping $\gamma$. The original edges of the parent triangles are also shown in grey. The relationship between the constituent and parent edges is shown in Fig.\ref{fig:EdgeRules}.}
        \label{fig:OrientedSub}                     
\end{figure}

From these mappings we can see why the explicit description of triangle orientation is necessary here. Otherwise, for instance, it would be ambiguous which constituent thick triangle, \textbf{T} or \textbf{T'}, we are describing inside a parent \textbf{T}.

We can alternatively visualize the mappings in Equations (\ref{eq:maps}) as directed graphs embedding constituent triangles into parent triangles, see Fig.\ref{fig:functionmap}. 

\begin{figure}[H]
        \begin{subfigure}[t]{0.6\textwidth}
                \includegraphics[width=\textwidth]{TRgraph}
        \end{subfigure}\hfill
        \begin{subfigure}[t]{0.4\textwidth}
                \includegraphics[width=\textwidth]{tsrgraph}
        \end{subfigure}\\
        
        \begin{subfigure}[t]{0.6\textwidth}
                \includegraphics[width=\textwidth]{TLgraph}
        \end{subfigure}\hfill
        \begin{subfigure}[t]{0.4\textwidth}
                \includegraphics[width=\textwidth]{tslgraph}
        \end{subfigure}  
        \caption{Relationship mapping from constituent triangles to parent triangles given by Equations (\ref{eq:maps})} 
        \label{fig:functionmap}                    
\end{figure}

We are now ready to introduce the Up process. Realizing that mappings from Equations (\ref{eq:maps}) will place constituent triangles inside parent triangles, we see that each application of a mapping will generate a larger finite region of the tiling. Think of this process as the opposite of substitution, we are starting from a constituent and generating a substituted parent triangle. Just as substitution generated a finite section of a tiling, by mapping a constituent to a substituted parent triangle we have similarly generated a finite section.

Further, just as with the substitution method, the end result of each mapping is an oriented Robinson triangle. That is, the image of every mapping will be the domain of other mappings. This allows us to form compositions of the mappings, from the image of one to the domain of another. To illustrate how these mappings can be composed from another, we have a cyclic directed graph in Fig.\ref{fig:UpDownGraph}.

\begin{figure}[H]
\centering
\includegraphics[width=\textwidth]{UpDownGraph}
\caption{A directed graph, or finite automaton, for the generation of parent triangles from constituent triangles. Paths along the directed edges represent compositions of the mappings from Equations (\ref{eq:maps}). Infinite paths along the directed edges may generate Penrose tilings of the plane.}
\label{fig:UpDownGraph}
\end{figure}

Paths along this directed graph represent compositions of the mappings. This is what is involved under the Up process, defining a sequence of the mappings. A finite sequence of mappings will directly correspond to a finite subregion of a Penrose tiling. The generation of the tiling region from the sequence of mappings is the Down process. Think of the Up process as producing a tiling instruction manual, and the Down process as following these instructions to produce a tiling. 

The Down process can describe a tiling region from the Up sequence because the mappings explicitly describe the relationship among substitution triangles. Even though each mapping only directly describes the relationship between a constituent triangle and its parent, we have also determined the placement of the other constituent oriented Robinson triangles of that parent. Further, the substitution rules from Fig.\ref{fig:OrientedRob} explicitly described the substitutions of the other constituent triangles. With this in mind, we can see how the Down process generates tiling regions from the Up process instructions. Essentially, the sequence of mappings from the Up process describes a sequence of substitutions, the Down process preforms these substitutions.

For an example of a finite sequence of the Up-Down process, consider the Up sequence from the graph path $\delta'\gamma\epsilon$ which defines the composition $\epsilon \circ \gamma \circ \delta'$. From the graph in Fig.\ref{fig:UpDownGraph} we can see that this Up sequence takes a \textbf{t} triangle and eventually embeds it into a \textbf{T} triangle. The Down process, given by this composition, then, will substitute that \textbf{T} triangle according to the composition sequence. This substitution ends once we've generated the initial \textbf{t} triangle. 


\begin{table}[H]
\begin{center}
\begin{tabular}{ccc}
\Large Up & & \Large Down\\
\raisebox{0px}{\includegraphics[width=0.3\textwidth]{Up4Left}} & \raisebox{20px}{\Large \textcolor{gray}{$ \rightarrow$}} & \includegraphics[width=0.6\textwidth]{Down4}\\
\Large $\uparrow$ $\epsilon$ & \hfill & $\downarrow$\\ 
\raisebox{0px}{\includegraphics[width=0.3\textwidth]{Up3Left}} & \raisebox{30px}{\Large \textcolor{gray}{$ \rightarrow$}} & \includegraphics[width=0.6\textwidth]{Down3}\\
\Large $\uparrow$ $\gamma$ & \hfill & $\downarrow$\\ 
\raisebox{0px}{\includegraphics[width=0.4\textwidth]{Up2Left}} & \raisebox{30px}{\Large \textcolor{gray}{$ \rightarrow$}} & \includegraphics[width=0.6\textwidth]{Down2}\\
\Large $\uparrow$ $\delta'$ & \hfill & $\downarrow$\\ 
\raisebox{0px}{\includegraphics[width=0.3\textwidth]{Up1Left}} &\raisebox{30px}{\Large \textcolor{gray}{$ \rightarrow$}}& \includegraphics[width=0.6\textwidth]{Down1}\\
\end{tabular}
\end{center}
\captionof{figure}{A finite region generated by the Up-Down process. The Up sequence $\delta'\gamma\epsilon$ describes the mapping composition $\epsilon \circ \gamma \circ \delta'$. The Up process maps a \textbf{t} triangle to a \textbf{T} triangle. The Down process applies those substitutions on a \textbf{T} triangle to generate the initial \textbf{t} triangle.}

\end{table}


Infinite sequences of the Up-Down process, corresponding to infinite paths along the directed graph, will generate Penrose tilings of the plane, half-plane, or $\frac{\pi}{5}$-plane wedge, depending on the sequence. Generating $\frac{\pi}{5}$-plane wedges are the result of repeated mappings into thin Robinson triangles, this occurs when the limit of the sequence generates a thin Robinson triangle. Effectively, wedges are when the `top' of the Up-Down process is an infinite \textbf{t} or \textbf{t'} triangle, and the Down process infinitely substitutes into this. Of course, under infinite sequences, the Up-Down process will have no `top' triangle, but we can see that certain arbitrarily long sequences can generate an arbitrarily large thin Robinson triangle tiling, and as such the limit of this process is the Penrose tiling of an infinite $\frac{\pi}{5}$-plane wedge. This is not a problem, however, in generating Penrose tilings of the entire plane, because these plane wedges can be extended to tilings of the plane by reflection along their edges. Reflections on the sides of wedges is valid under the edge matching rules, to illustrate this with finite thin Robinson triangles see Fig.\ref{fig:WedgeWheel}.  Penrose plane tilings generated by $\frac{\pi}{5}$-plane wedge reflection will exhibit fivefold rotational symmetry centered at the point of the wedge. As we will discuss later, five-fold rotational symmetry is only possible in aperiodic tilings, so this is a remarkable feature of Penrose tilings generated from plane wedges. Similarly, half-plane tilings occur when the limit of the process is an arbitrarily large thick Robinson triangle. Half-plane tilings can also be extended to full plane tilings by reflection across the base.

\begin{figure}
\centering
\includegraphics[width=0.5\textwidth]{WedgeWheel}
\caption{Reflection along sides of thin Robinson triangles produces valid arrangement of triangles under the edge matching rules. Demonstrated for finite triangles, this is also true for infinite  $\frac{\pi}{5}$-plane wedges. As such,  $\frac{\pi}{5}$-plane wedges can be extended to tilings of the entire plane under side reflection. Planes tiled in this way will exhibit fivefold rotational symmetry about the wedge point.}
\label{fig:WedgeWheel}
\end{figure}

\subsection{Penrose Tilings as Infinite Sequences}
We've seen how the Up-Down method generates Penrose tilings from infinite compositions of mappings. By describing these infinite compositions as sequences, we've alluded to an alternative representation of a Penrose tiling, as an infinite sequence of steps on the directed graph in Fig.\ref{fig:UpDownGraph}. 

\begin{mydef}
A \textbf{composition sequence}, $\rho$, is an infinite sequence of mappings which, under the Up-Down process, will take an elementary Robinson triangle to a Penrose tiling of the plane, half-plane, or $\frac{\pi}{5}$-plane wedge. Compositions sequences must correspond to infinite paths on the graph in Fig.\ref{fig:UpDownGraph}.
\end{mydef}

\begin{mylem}
Every composition sequence generates a Penrose tiling of a plane, half-plane, or $\frac{\pi}{5}$-plane wedge.
\end{mylem}

\begin{proof}
This follows directly from the Up-Down method.
\end{proof}

By the above lemma, we see that every composition sequence generates a Penrose tiling. However, Penrose tilings are not unique to a composition sequence.

\begin{mylem}
A Penrose tiling is generated by infinitely many composition sequences.
\end{mylem}

\begin{proof}
Consider a Penrose tiling, $\mathcal{T}$.\\
$\mathcal{T}$ is composed of infinitely many individual tiles.\\
By the Up-Down method, each tile can generate $\mathcal{T}$ by a unique composition sequence.\\
$\therefore$ There are infinitely many composition sequences which generate $\mathcal{T}$.
\end{proof}

We've now shown that while all composition sequences correspond to a Penrose tiling, this relationship isn't one-to-one. That is, we see that infinitely many sequences generate the same Penrose tiling. What determines whether the same Penrose tiling is generated by two different sequences?

\begin{mydef}
Two composition sequences, $\rho_1$ and $\rho_2$, are said to be \textbf{cofinal} if they agree after finitely many terms.
\end{mydef}

\begin{mythm}
Two composition sequences generate the same Penrose tiling if and only if they are cofinal.
\end{mythm}

\begin{proof}
Let $\rho_1$ and $\rho_2$ be composition sequences defining $\mathcal{T}_1$ and $\mathcal{T}_2$ respectively. \\[10pt]
If $\rho_1 = \rho_2$ then $\mathcal{T}_1 = \mathcal{T}_2$, and they are trivially cofinal.\\[10pt]
If $\rho_1$ and $\rho_2$ are cofinal, then, after, $n$, finitely many terms, the sequences agree.\\
Consider composition sequence, $\rho^*$, which begins where $\rho_1$ and $\rho_2$ agree.\\
$\rho^*$ defines a Penrose tiling  $\mathcal{T}^*$.\\
This implies that, at some hierarchal level, level $n$ in the Up process, $\rho_1$ and $\rho_2$ begin describing the same tiling.\\
By the Up-Down process, substitution is uniquely defined.\\
Therefore, $\mathcal{T}^*$ uniquely determines $\mathcal{T}_1$ after $n$ substitutions.\\
Likewise, $\mathcal{T}^*$ uniquely determines $\mathcal{T}_2$ after $n$ substitutions.\\
$\therefore$ $\mathcal{T}_1 = \mathcal{T}_2$ if  $\rho_1$ and $\rho_2$ are cofinal.\\[10pt]
Conversely, suppose $\mathcal{T}_1 = \mathcal{T}_2$ and call this tiling $\mathcal{T}$. \\
Let $\rho_1$ begin with the tile  ${\rho_1}_0$.\\
Let $\rho_2$ begin with the tile ${\rho_2}_0$.\\
Then ${\rho_1}_0$ and ${\rho_2}_0$ both belong to $\mathcal{T}$.\\
The distance between ${\rho_1}_0$ and ${\rho_2}_0$ is therefore finite.\\
Then at some hierarchal level, after finite $n$ in the Up process, a single tile will contain both ${\rho_1}_0$ and ${\rho_2}_0$.\\
$\rho_1$ and $\rho_2$ will agree after this level.\\
$\therefore$ $\rho_1$ and $\rho_2$ are cofinal if $\mathcal{T}_1$ and $\mathcal{T}_2$.
\end{proof}

We now see that all cofinal sequences are related to each other by the unique Penrose tiling they generate. This defines an equivalence class.

\begin{mydef}
Cofinality of composition sequences defines an equivalence relation on the set of composition sequences. Equivalence classes of cofinal sequences are called \textbf{families}.
\end{mydef}

Further, we can consider a function, $C(\rho)$, which takes a composition sequence and returns its parent sequence. In other words, if $\rho$ defines a Penrose tiling, $C(\rho)$ defines the tiling before the previous substitution. 

\begin{mylem}
$\rho$ and $\rho'$ are cofinal if and only if $C(\rho)$ and $C(\rho')$ are cofinal.
\end{mylem}

\begin{proof}
Let 
\begin{equation*}\begin{split}
\rho=\{\rho_0,\rho_1,\rho_2,\rho_3,\mathellipsis\}\\
\rho'=\{\rho'_0,\rho'_1,\rho'_2,\rho'_3,\mathellipsis\}
\end{split}
\end{equation*}\\
Then 
\begin{equation*}
\begin{split}
C(\rho)=\{\rho_1,\rho_2,\rho_3,\mathellipsis\}\\
C(\rho')=\{\rho'_1,\rho'_2,\rho'_3,\mathellipsis\}
\end{split}
\end{equation*}\\
By definition of cofinality and inspection, $\rho$ and $\rho'$ are cofinal if and only if $C(\rho)$ and $C(\rho')$ are cofinal.
\end{proof}

Further, $C(\rho)$ allows us to address a common misconception regarding Penrose tilings. When looking at a graphical representation of a tiling, especially when considering their construction by substitution, it is easy to assume that the Penrose tiling is self-similar. We can easily show this is not the case in general.

\begin{mylem}
Penrose tilings are, in general, not self-similar.
\end{mylem}

\begin{proof}
Consider a Penrose tiling, $\mathcal{T}$.\\
$\mathcal{T}$ is given by composition sequence $\rho$.\\
In general, $C(\rho)\neq\rho$.\\
$\therefore$ Penrose tilings are not self-similar.
\end{proof}

However, we arrive at something of a veridical paradox here. We've just shown that Penrose tilings are not self-similar. However, we also know that, trivially, $\rho$ and $C(\rho)$ are cofinite. Therefore, we see that $\rho$ and $C(\rho)$ define the same Penrose tiling. That is, substitution of a Penrose tiling will map the tiling back to itself, but this will not be self-similar. 

Finally, we are ready to see the greatest benefit to considering Penrose tilings as equivalence classes of composition sequences.

\begin{mythm}
There are uncountably many Penrose tilings.
\end{mythm}

\begin{proof}
Every Penrose tiling is uniquely determined by a family of cofinal composition sequences.\\
By Cantor's Diagonalization Argument, if there were countably many Penrose tilings then we could construct a countable list of infinite sequences, with one sequence from each family.\\
However, by considering infinite paths on the directed graph or recalling the proof of the uncountability of real numbers, we see that given such a countable list we can always construct a composition sequence that is not a member of that list.\\
Therefore, a countable list of composition sequence families will always be incomplete.\\
Therefore, there are uncountably many composition sequence families.\\
$\therefore$ there are uncountably many Penrose tilings by Cantor's Diagonalization.
\end{proof}

\section{Projective Construction}


\chapter{Features of the Penrose Tiling}
\section{Mutual Local Derivability of Representations}
\section{Five-Fold Rotational Symmetry}
\begin{mythm}
No tiling can have more than one center of fivefold rotational symmetry.
\label{symthm}
\end{mythm}

\begin{figure}[H]
	\centering
	\includegraphics[width=0.4\textwidth]{proof}
    \caption{Barlow's Proof that no tiling can have more than one center of fivefold symmetry.}
    \label{fig:fivefoldproof}
\end{figure}

Conway provides a proof of this, attributed to Peter Barlow \cite{Gardner1997}.

\begin{proof}
Suppose in order to derive a contradiction that a tiling $\mathcal{T}$ has more than one center of fivefold symmetry.\\
Choose two centers, A and B, such that they are the nearest possible centers. \\
Rotate $\mathcal{T}$ counter-clockwise by $\frac{2\pi}{5}$ about $A$, bringing $B$ to $B'$, (see Fig.\ref{fig:fivefoldproof}).\\
Rotate $\mathcal{T}$ clockwise by $\frac{2\pi}{5}$ about $B$, bringing $A$ to $A'$.\\
A and B are centers of fivefold symmetry, so the tiling will overlap identically.\\
$\implies$ $A'$ and $B'$ are centers of fivefold symmetry.\\
But $A'$ and $B'$ are closer than $A$ and $B$.\\
This contradicts hypothesis that $A$ and $B$ are nearest centers of fivefold symmetry.\\
$\therefore$ By contradiction, $\mathcal{T}$ can only have one center of fivefold symmetry.
\end{proof}

We've shown that no tiling can have more than one center of fivefold symmetry. Using this theorem we can show, further, that even one center of fivefold symmetry is impossible if the tiling is periodic. First, consider these properties of periodic tilings:

\begin{myprop}
A periodic tiling can be subdivided into countably infinitely many regions, $\mathcal{R}$, which are congruent by translation such that
\begin{enumerate}
\item Two regions have no common interior points: $\mathring{\mathcal{R}_i} \cap  \mathring{\mathcal{R}_j}=\emptyset$ if $i\neq j$
\item The union of regions is exactly the tiling: $\bigcup_{i=1}^\infty \mathcal{R}_i = \mathcal{T}=\mathbb{E}^n$
\item For any two regions, $\mathcal{R}_i$ and $\mathcal{R}_j$, there is a translation that takes  $\mathcal{R}_i$ to $\mathcal{R}_j$. This translation also admits symmetry, carrying the entire tiling onto itself. \label{periodprop3}
\end{enumerate}
\label{periodprop}
\end{myprop}

\begin{mylem}
If tiling $\mathcal{T}$ is periodic, then it cannot have any centers of fivefold symmetry.
\end{mylem}
The proof of this is identical to the proof of Theorem \ref{symthm}.
\begin{proof}
Suppose in order to derive a contradiction that A is a center of fivefold rotational symmetry.\\
Since $\mathcal{T}$ is periodic, by condition \ref{periodprop3} of Property \ref{periodprop}, we see that there must be countably infinitely many centers of fivefold symmetry.\\
By Theorem \ref{symthm}, this is impossible.\\
$\therefore$ By contradiction, periodic tilings have no centers of fivefold symmetry.
\end{proof}

We have shown that fivefold symmetry is incompatible with periodic tilings. For this reason, the admittance of a single center fivefold symmetry is related to the aperiodicity of a tiling. As we will see, centers of fivefold symmetry are possible in Penrose tilings. 
\section{Worms}
\section{Local Isomorphism}
\section{Ratio of Prototiles}
\chapter{Graphical Geodesic Paths}
\section{Path Generation Rules}
\section{Types of Paths}

\bibliographystyle{unsrt}
\bibliography{bibliography}    
\end{document}