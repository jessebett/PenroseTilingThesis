\documentclass[10pt, compress]{beamer}

\usetheme[usetitleprogressbar]{m}

\usepackage{booktabs}
\usepackage[scale=2]{ccicons}
\usepackage{minted}
\usepackage{pgfplots}
\usepackage{graphicx, float}
\usepackage{caption}
\usepackage{subcaption}
\usepackage{textpos}

\newcommand<>{\fullsizegraphic}[1]{
  \begin{textblock*}{0cm}(-1cm,-3.78cm)
  \includegraphics[width=\paperwidth]{#1}
  \end{textblock*}
}


\graphicspath{{./images/}}

%\usepgfplotslibrary{dateplot}

\usepackage{amsthm}
\newtheorem{mydef}{\alert{Definition:}}
\newtheorem{ex}{\alert{Example:}}

\usemintedstyle{trac}

\title{\huge The Beautiful Geometry of Aperiodic Tessellations}
\subtitle{}
\date{\today}
\author{Jesse Bettencourt}
\institute{Supervised by Miroslav Lovric}

\begin{document}

\maketitle
\section{What is a Tessellation?}

\begin{frame}{Figures}
  \frametitle{A Simple Tessellation}
  
  \begin{figure}
	\includegraphics[width=0.5\textwidth]{Checkerboard}
    \caption{Checkerboard Tessellation}
    \label{fig:checkerboard}
  \end{figure}
  
  
\end{frame}

\begin{frame}[fragile]
  \frametitle{What is a Tessellation?}
\begin{mydef}
A \textbf{tessellation} $\mathcal{T}$ of the space $\mathbb{E}^n$ is a countable family of closed sets, $T$, called \emph{tiles}:
\begin{equation*}
\mathcal{T}=\{T_1,T_2,\mathellipsis\}
\end{equation*}
such that
\begin{enumerate}
\item $\mathcal{T}$ has \textbf{no overlaps}: $\mathring{T_i} \cap \mathring{T_j}=\emptyset$ if $i\neq j$
\item $\mathcal{T}$ has \textbf{no gaps}: $\bigcup_{i=1}^\infty T_i = \mathbb{E}^n$
\end{enumerate}
\end{mydef}

\begin{center}
\includegraphics[width=0.2\textwidth]{Checkerboard}
\end{center}
\end{frame}



\begin{frame}[fragile]
\frametitle{Components of a Tessellation}

\begin{mydef}
Let $\{T_1,T_2,\mathellipsis\}$ be the set of tiles of tessellation $\mathcal{T}$, partitioned into a set of equivalence classes by \textbf{criterion} $\mathcal{M}$. The set, $\mathcal{P}$, of representatives of these equivalence classes is called the \textbf{protoset} for $\mathcal{T}$ with respect to $\mathcal{M}$.
\end{mydef}

\begin{ex}
\begin{table}
\begin{tabular}{l c}
  \textbf{Criterion:} & $\mathcal{M}=$ \{\emph{Colour of the tile. Only opposite colours may touch.}\}\\[10pt]
  \textbf{Protoset:} &  
     \huge  \raisebox{0pt}{$\mathcal{P}=$ \{} 
                  \raisebox{-10pt}{\includegraphics[width=0.1\textwidth]{tealproto}}\raisebox{0pt}{ ,}
                   \raisebox{-10pt}{\includegraphics[width=0.1\textwidth]{whiteproto}}
          \huge\raisebox{0pt}{\}}\\
\end{tabular}
\end{table}
\end{ex}

\end{frame}


\begin{frame}[fragile]
\frametitle{Protosets Admit Tessellations}
\begin{mydef}
If $\mathcal{T}$ is a tessellation with protoset $\mathcal{P}$, then we say that $\mathcal{P}$ admits $\mathcal{T}$.
\end{mydef}


\begin{ex}
\flushleft We say\\
\centering\huge \raisebox{0pt}{$\mathcal{P}=$ \{} 
                  \raisebox{0pt}{\includegraphics[width=0.05\textwidth]{tealproto}}\raisebox{0pt}{ ,}
                   \raisebox{0pt}{\includegraphics[width=0.05\textwidth]{whiteproto}}
          \huge\raisebox{0pt}{\}}\\
\flushleft\normalsize \raisebox{15pt}{admits}\\
\centering\includegraphics[width=0.2\textwidth]{Checkerboard}
\end{ex}
\end{frame}

\begin{frame}{Figures}
\frametitle{Protosets Can Admit Multiple Tessellations}
\centering\huge \raisebox{0pt}{$\mathcal{P}=$ \{} 
                  \raisebox{0pt}{\includegraphics[width=0.05\textwidth]{tealproto}}\raisebox{0pt}{ ,}
                   \raisebox{0pt}{\includegraphics[width=0.05\textwidth]{whiteproto}}
          \huge\raisebox{0pt}{\}}\\
\flushleft\normalsize \raisebox{15pt}{admits both}\\

\begin{figure}
\begin{subfigure}[b]{0.4\textwidth}
\includegraphics[width=\textwidth]{Checkerboard}
\end{subfigure}\hfill \raisebox{60pt}{and} \hfill
\begin{subfigure}[b]{0.4\textwidth}
\includegraphics[width=\textwidth]{CheckerboardOther}
\end{subfigure}
\end{figure}
\end{frame}

\begin{frame}{Figures}
\frametitle{Matching Rules Correspond to Deformed Protosets}
\centering
Edge deformations can force matching rules.

\centering\huge \raisebox{0pt}{$\mathcal{P}=$ \{} 
                  \raisebox{0pt}{\includegraphics[width=0.05\textwidth]{tealproto}}\raisebox{0pt}{ ,}
                   \raisebox{0pt}{\includegraphics[width=0.05\textwidth]{whiteproto}}
          \huge\raisebox{0pt}{\}}\hfill
          \huge \raisebox{0pt}{$\mathcal{P}=$ \{} 
                            \raisebox{0pt}{\includegraphics[width=0.05\textwidth]{protodisk}}\raisebox{0pt}{ ,}
                             \raisebox{0pt}{\includegraphics[width=0.05\textwidth]{protosquare}}
                    \huge\raisebox{0pt}{\}}\\
\begin{figure}
\begin{subfigure}[b]{0.3\textwidth}
\includegraphics[width=\textwidth]{Checkerboard}
\end{subfigure} \raisebox{40pt}{ $\rightarrow$} 
\begin{subfigure}[b]{0.3\textwidth}
\includegraphics[width=\textwidth]{CheckerboardCircles}
\end{subfigure}
\end{figure}

\normalsize Matching Rules $\implies$ Deformed Protoset
\end{frame}

\section{Describing Tessellations}

\begin{frame}
\frametitle{Symmetries}

\begin{mydef}
A tessellation is said to be \textbf{symmetric} under a transformation if that transformation maps the tiling to itself identically.
\end{mydef}

\alert{Rotational Symmetry:} $\mathcal{T}$ can be rotated a non-trivial angle about a point and overlap itself identically\\[10pt]
\alert{Translational Symmetry:} $\mathcal{T}$ can be shifted by some non-trivial distance in a direction and overlap itself identically.
\end{frame}

\begin{frame}{Figures}
\frametitle{Checkerboard Symmetries}
\begin{figure}
\begin{subfigure}[b]{0.45\textwidth}
\includegraphics[width=\textwidth]{Checkerboard}
\caption*{Translation Symmetry: 2 Squares\\ Rotational Symmetry: $\frac{\pi}{2}$}
\end{subfigure}\hfill
\begin{subfigure}[b]{0.45\textwidth}
\includegraphics[width=\textwidth]{CheckerboardOther}
\caption*{Translation Symmetry: 3 Squares\\ Rotational Symmetry: $\pi$}
\end{subfigure}
\end{figure}
\end{frame}

\begin{frame}
\frametitle{Periodicity}
\begin{mydef}
A tessellation is said to be \textbf{periodic} if it admits translational symmetry in two directions.
\end{mydef}
\begin{mydef}
A tessellation is said to be \textbf{non-periodic} if it admits no translational symmetries.
\end{mydef}
\end{frame}

\begin{frame}{Figures}
\frametitle{Checkerboard Periodicity}
\begin{figure}
\begin{subfigure}[b]{0.45\textwidth}
\includegraphics[width=\textwidth]{Checkerboard}
\caption*{Periodic}
\end{subfigure}\hfill
\begin{subfigure}[b]{0.45\textwidth}
\includegraphics[width=\textwidth]{CheckerboardAperiodic}
\caption*{Non-Periodic}
\end{subfigure}
\end{figure}
\end{frame}

\begin{frame}
\frametitle{The Aperiodic Question}

\centering
Checkerboard protoset,\\
\huge \raisebox{0pt}{$\mathcal{P}=$ \{} 
                  \raisebox{0pt}{\includegraphics[width=0.05\textwidth]{tealproto}}\raisebox{0pt}{ ,}
                   \raisebox{0pt}{\includegraphics[width=0.05\textwidth]{whiteproto}}
          \huge\raisebox{0pt}{\}}\\ \normalsize admits \alert{both} \textbf{periodic} and \textbf{non-periodic} tessellations.\\[15pt]


\begin{figure}
\begin{subfigure}[b]{0.35\textwidth}
\includegraphics[width=\textwidth]{Checkerboard}
\caption*{Periodic}
\end{subfigure}\hfill
\begin{subfigure}[b]{0.35\textwidth}
\includegraphics[width=\textwidth]{CheckerboardAperiodic}
\caption*{Non-Periodic}
\end{subfigure}
\end{figure}

\end{frame}

\begin{frame}
\frametitle{The Aperiodic Question}

\centering
Checkerboard protoset,\\
\huge \raisebox{0pt}{$\mathcal{P}=$ \{} 
                  \raisebox{0pt}{\includegraphics[width=0.05\textwidth]{tealproto}}\raisebox{0pt}{ ,}
                   \raisebox{0pt}{\includegraphics[width=0.05\textwidth]{whiteproto}}
          \huge\raisebox{0pt}{\}}\\ \normalsize admits \alert{both} \textbf{periodic} and \textbf{non-periodic} tessellations.\\[15pt]

Some protosets,\\
\huge \raisebox{0pt}{$\mathcal{P}=$ \{} 
                  \raisebox{0pt}{\includegraphics[width=0.08\textwidth]{HexagonProto}}\raisebox{0pt}
          \huge\raisebox{0pt}{\}}\\ \normalsize admits \alert{only} \textbf{periodic} tessellations.\\[15pt]

\begin{figure}
\includegraphics[width=0.2\textwidth]{Hexagons}
\end{figure}

\invisible{
\begin{figure}
\begin{subfigure}[b]{0.35\textwidth}
\includegraphics[width=\textwidth]{Checkerboard}
\caption*{Periodic}
\end{subfigure}\hfill
\begin{subfigure}[b]{0.35\textwidth}
\includegraphics[width=\textwidth]{CheckerboardAperiodic}
\caption*{Non-Periodic}
\end{subfigure}
\end{figure}
}
\end{frame}

\begin{frame}
\frametitle{The Aperiodic Question}

\centering
Checkerboard protoset,\\
\huge \raisebox{0pt}{$\mathcal{P}=$ \{} 
                  \raisebox{0pt}{\includegraphics[width=0.05\textwidth]{tealproto}}\raisebox{0pt}{ ,}
                   \raisebox{0pt}{\includegraphics[width=0.05\textwidth]{whiteproto}}
          \huge\raisebox{0pt}{\}}\\ \normalsize admits \alert{both} \textbf{periodic} and \textbf{non-periodic} tessellations.\\[15pt]

Some protosets,\\
\huge \raisebox{0pt}{$\mathcal{P}=$ \{} 
                  \raisebox{0pt}{\includegraphics[width=0.08\textwidth]{HexagonProto}}\raisebox{0pt}
          \huge\raisebox{0pt}{\}}\\ \normalsize admits \alert{only} \textbf{periodic} tessellations.\\[15pt]
          
Are there any protosets,\\
\huge \raisebox{0pt}{$\mathcal{P}=$ \{?\}}\\ 
\normalsize that admit \alert{only} \textbf{non-periodic} tessellations?
\invisible{
\begin{figure}
\includegraphics[width=0.2\textwidth]{Hexagons}
\end{figure}
}

\invisible{
\begin{figure}
\begin{subfigure}[b]{0.35\textwidth}
\includegraphics[width=\textwidth]{Checkerboard}
\caption*{Periodic}
\end{subfigure}\hfill
\begin{subfigure}[b]{0.35\textwidth}
\includegraphics[width=\textwidth]{CheckerboardAperiodic}
\caption*{Non-Periodic}
\end{subfigure}
\end{figure}
}


\begin{figure}
\includegraphics[width=0.2\textwidth]{Hexagons}
\end{figure}

\invisible{
\begin{figure}
\begin{subfigure}[b]{0.35\textwidth}
\includegraphics[width=\textwidth]{Checkerboard}
\caption*{Periodic}
\end{subfigure}\hfill
\begin{subfigure}[b]{0.35\textwidth}
\includegraphics[width=\textwidth]{CheckerboardAperiodic}
\caption*{Non-Periodic}
\end{subfigure}
\end{figure}
}
\end{frame}

\section{The Penrose Tessellation}

\begin{frame}
\frametitle{The Penrose Rhombs Protoset}
\centering
\huge \raisebox{25pt}{$\mathcal{P}=$ \{} 
                  \raisebox{0pt}{\includegraphics[width=0.3\textwidth]{FatProto}}\raisebox{20pt}{ ,}
                   \raisebox{-10pt}{\includegraphics[width=0.1\textwidth]{SkinnyProto}}
          \huge\raisebox{25pt}{\}*}
          
\vfill \raisebox{-7pt}{*} \normalsize With complicated matching rules
\end{frame}

\begin{frame}
\frametitle{Constructing the Penrose Tessellation}
The Penrose Tessellation cannot be constructed \textbf{locally}, instead we use \textbf{substitution rules}:

\begin{table}
\begin{tabular}{lcr}
\raisebox{15pt}{\includegraphics[width=0.2\textwidth]{FatProto}} & \Huge \raisebox{30pt}{$\rightarrow$} & \includegraphics[width=0.2\textwidth]{FatSub} \\
\raisebox{10pt}{\includegraphics[width=0.08\textwidth]{SkinnyProto}} & \Huge \raisebox{40pt}{$\rightarrow$} & \includegraphics[width=0.2\textwidth]{SkinnySub} \\
\end{tabular}
\end{table}

\end{frame}

\begin{frame}
\frametitle{Start with a Seed: Fat Rhombus}
\begin{figure}
\includegraphics[width=0.6\textwidth]{FatProto}
\end{figure}

\end{frame}

\begin{frame}
\frametitle{Apply Substitution Rules}
\begin{figure}
\includegraphics[width=0.6\textwidth]{Inflate1}
\end{figure}

\end{frame}

\begin{frame}
\frametitle{Apply Substitution Rules Again}
\begin{figure}
\includegraphics[width=0.6\textwidth]{Inflate2}
\end{figure}

\end{frame}

\begin{frame}
\frametitle{Apply Substitution Rules Again and Again}
\begin{figure}
\includegraphics[width=0.6\textwidth]{Inflate3}
\end{figure}

\end{frame}

\begin{frame}
\frametitle{Apply Substitution Rules Again and Again and Again}
\begin{figure}
\includegraphics[width=0.6\textwidth]{Inflate4}
\end{figure}

\end{frame}

\begin{frame}
\frametitle{Apply Substitution Rules Again and Again and Again and Again}
\begin{figure}
\includegraphics[width=0.6\textwidth]{Inflate5}
\end{figure}

\end{frame}

\begin{frame}
\frametitle{Apply Substitution Rules 10 times}
\begin{figure}
\includegraphics[width=0.9\textwidth]{Inflate10}
\end{figure}

\end{frame}

\begin{frame}
\frametitle{Apply Substitution Rules 11 times}

  \fullsizegraphic{Inflate11}


\end{frame}

\end{document}
